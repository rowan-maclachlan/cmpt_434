\documentclass[10pt]{amsart}
\usepackage{hyperref}
\usepackage{parskip,fullpage}
\usepackage{graphicx}
\usepackage{amsmath}
\usepackage{amsthm}
\usepackage[T1]{fontenc}
\usepackage{geometry}
\usepackage{siunitx}
\sisetup{output-exponent-marker=\ensuremath{\mathrm{e}}}

\geometry{
	body={7in, 9.5in},
	left=.5in,
	top=.50in
}

\newtheorem*{theorem}{Theorem}
\newtheorem{lemma}{Lemma}
\calclayout
\begin{document}
\textbf{CMPT 434 Assignment 1} \\
Winter 2019\\
Due Date: Saturday January 25th\\
rdm659 11165820

Part B

\begin{enumerate}
    \item (4 marks) Consider two nodes connected by a single dedicated link within an OCN (on- chip network), SAN (system/storage area network), LAN (local area network), or WAN (wide-area network), and suppose that we wish to transmit a single 100 byte (including header) packet from one node to the other. Calculate the total delay and the percentage of the total delay that is propagation delay for each network, assuming that:
    \begin{itemize}
        \item the link data rate is 4 Gbps (i.e., $4 x 109$ bps);
        \item there is no queuing delay;
        \item the total node processing delay (not overlapped with any other
            component of delay) is x + (0.5 nanoseconds/byte), where x (the
            portion of this delay that is independent of packet size) is 0
            microseconds for the OCN, 0.3 microseconds for the SAN, 3
            microseconds for the LAN, and 30 microseconds for the WAN; and
        \item the link distances are 0.5 cm, 5m, 5000m, and 5000 km, for the
            OCN, SAN, LAN, and WAN, respectively, with the speed of signal
            propagation in each case equal to 200,000 km/s (approximately 2/3 of
            the speed of light in a vacuum).
    \end{itemize}

    (Assuming the given measurements are infinitely precise.)
    \begin{enumerate}
        \item What is the node processing delay?
        \begin{enumerate}
            \item On the OCN?\\
                $NPD_{OCN} = 100b * (0.5ns/b) = 50ns = \num{5.0e-8}s$
            \item On the SAN?\\
                $NPD_{SAN} = 0.3ns + (100b * 0.5ns/b) = 50.3ns = \num{5.3e-8}s$
            \item On the LAN?\\
                $NPD_{LAN} = 3\mu s + (100b * 0.5ns/b) = 3\mu s + 50ns = \num{3.05e-6}s$
            \item On the WAN?\\
                $NPD_{WAN} = 30\mu s + (100b * 0.5ns/b) = 30\mu s + 50ns = \num{3.005e-6}s$
        \end{enumerate}
        
        \item What is the propagation delay?
        \begin{enumerate}
            \item On the OCN?\\
                $PD_{OCN} = 0.005m / 200,000,000\si{\meter\per\second} = \num{2.5e-11}s$
            \item On the SAN?\\
                $PD_{SAN} = 5m / 200,000,000\si{\meter\per\second} = \num{2.5e-8}s$
            \item On the LAN?\\
                $PD_{LAN} = 5,000m / 200,000,000\si{\meter\per\second} = \num{2.5e-5}s$
            \item On the WAN?\\
                $PD_{WAN} = 5,000,000m / 200,000,000\si{\meter\per\second} = \num{2.5e-2}s$
        \end{enumerate}
        
        \item What is the percentage of the total delay which is propagation delay?
        \begin{enumerate}
            \item On the OCN?\\
                $OCN = PD_{OCN} / (PD_{OCN} + NPD_{OCN}) = \num{2.5e-11}s / (\num{2.5e-11}s
                    + \num{5.0e-8}s) \approx 0.05\%$
            \item On the SAN?\\
                $SAN = PD_{SAN} / (PD_{SAN} + NPD_{SAN}) = \num{2.5e-8}s / (\num{2.5e-8}s +
                    \num{5.3e-8}s) \approx 32\%$
            \item On the LAN?\\
                $LAN = PD_{LAN} / (PD_{LAN} + NPD_{LAN}) = \num{2.5e-5}s / (\num{2.5e-5}s +
                    \num{3.05e-6}s) \approx 89\%$
            \item On the WAN?\\
                $WAN = PD_{WAN} / (PD_{WAN} + NPD_{WAN}) = \num{2.5e-2}s / (\num{2.5e-2}s +
                    \num{3.005e-6}s) \approx 100\%$
        \end{enumerate}
    \end{enumerate}

    \item (4 marks) Consider a message consisting of n packets, each of 1000
        bytes (including header), that traverses 4 links in a store-and-forward
        packet-switched network (i.e., there are 3 intermediate nodes between
        the source and destination that the message passes through). The
        propagation delay on each link is 1 ms. The transmission rate on each
        link is 100 Mbps (i.e., 100 x 106 bps). Neglecting processing and
        queueing delays, and assuming that the packets are sent back-to-back
        without intervening delays, give the total delay from when the first bit
        is sent at the source until all packets have been completely received at
        the destination for the following values of n:
    \begin{enumerate}
        \item $n = 1$\\
        The propagation delay between the sender and receiver on this network is $4 * 1ms = 4ms = 0.004s$.\\
        Total message size is $1000 bytes * 8bits/byte = 8000 bits$.\\
        The transmission rate is $100Mbps = 100000000bps$\\
        Then it takes $8000b / 100000000bps = \num{8e-5}s$ to get the last bit on the wire.\\
        Consider the last bit sent from the receiver.  It get on the wire after
            $\num{8e-5}s$.  It takes $1ms$ to get to the next node.\\
        This is repeated 4 times before the last bit is received at its
            destination.  Then, the total delay is $4 * (\num{8e-5}s +
            \num{1e-3}s) = \num{4.32e-3}s$
        \item $n = 12$\\
        If the total amount of data is increased 12 fold, then we should expect
            the only thing to change is the total message size.\\
        Now, the total message size is $1000bytes/message * 8 bits/byte * 12 messages = 96000bits$.\\
        Then, it takes $96000b / 100000000bps = \num{9.6e-4}s$ for the last bit
            to get on the wire, and the total delay will then be $4 *
            (\num{9.6e-4}s + \num{1e-3}s) = \num{7.84e-3}s$
    \end{enumerate}

    \item (4 marks) Give the maximum data rate, as measured in Kbps, for transmissions on a channel with bandwidth H = 400 KHz and a signal to noise ratio of 63, for each of the following two cases. (Hint: for each case, use whichever of Nyquist’s theorem and Shannon’s theorem imposes the tightest constraint.)
    \begin{enumerate}
        \item Each transmitted symbol has 2 possible values (i.e., transmits just a single bit)\\
            According to Nyquist's Theorem, the maximum symbol rate is $baud =
            2H = 800Kbps$ symbols per second.  Then, the baud is also the
            maximum data rate because there is only 1 bit per symbol.\\
            According to Shannon's Theorem, the maximum bps is
            $400,000Hz*\log{2}{1 + 63}$ which gives $bps_{max} = 400,000*6 =
            2,400Kbps$\\ In this case, Nyquist's Theorem places the greatest
            constraint on the maximum data rate.\\
        \item Each transmitted symbol has 32 possible values.\\
            With Nyquist's Theorem, the maximum symbol rate is $baud = 2H =
            800K$ symbols per second.  Then, the $bps_{max} = 800K * 32 =
            25,600Kbps$\\
            With to Shannon's Theorem, we find the same result, and the
            $bps_{max} = 2,400Kbps$\\ In this case, Shannon's Theorem places the
            greatest constraint on the maximum data rate.\\
    \end{enumerate}
    \item (4 marks) Consider a scenario in which 100 sessions share a link of data
    rate 100 Mbps (i.e., 100 x $10^6$ bps). Suppose that each session alternates
    between idle periods (when no data is being sent) of average duration 1
    second, and busy periods of average duration B.
    \begin{enumerate}
        \item Supposing that circuit switching is used (i.e., channel capacity
            is evenly divided among the sessions, with a session’s allocated
            capacity unused when the session is idle), give the achievable data
            transmission rate a session can achieve during its busy period.\\
            If there are a hundred sessions on the 100 Mbps link, then each
            sessions 1/100 of that capacity.  Therefore, a specific session has
            1 Mbps data rate.  During its busy period, a session could
            theoretically use the entire 1 Mpbs to acheive a 1 Mpbs data
            transmission rate.
        \item Give an estimate of the average achievable data rate of each
            session (during its busy period) when packet switching is used, for
            B = 10 seconds, B = 1 second, B = 100 milliseconds, and B = 10
            milliseconds, assuming:
        \begin{itemize}
            \item at each point in time the link capacity is shared approximately
                equally among all of the sessions currently in their busy period; and
            \item when a given session is in its busy period, the number of other busy
                sessions can be approximated by the total number of other sessions (99)
                times the fraction of time that each is in its busy period.
        \end{itemize}
        \begin{enumerate}
            \item $B = 10s$\\
                When $B = 10s$, then the average bandwidth for a session
                ($BW_{avg}$) in its busy period will be around $100 Mbps / 1 +
                (99 * ( B / I + B )$ where $I$ is the idle period of 1 second.
                Then, \[ BW_{avg} = 100 Mbps / 1 + ( 99 * ( 10s / 11s)) \]
                      \[ BW_{avg} = 100 Mbps / 91 \approx 1.1 Mbps \]
            \item $B = 1s$\\
                \[ BW_{avg} = 100 Mbps / 1 + ( 99 * ( 1s / 2s)) \]
                \[ BW_{avg} = 100 Mbps / 50.5 \approx 2 Mbps \]
            \item $B = 100ms$\\
                \[ BW_{avg} = 100 Mbps / 1 + ( 99 * ( \num{1e-1}s / \num{1.1e0}s)) \]
                \[ BW_{avg} = 100 Mbps / 1 + ( 99 * \num{1e1}s\]
                \[ BW_{avg} = 100 Mbps / 10 \approx 10 Mbps \]
            \item $B = 10ms$\\
                \[ BW_{avg} = 100 Mbps / 1 + ( 99 * ( \num{1e-2}s / \num{1.01e-0}s)) \]
                \[ BW_{avg} = 100 Mbps / 1 + ( .98) \]
                \[ BW_{avg} = 100 Mbps / 1.98 \approx 102 Mbps \]
        \end{enumerate}
    \end{enumerate}
    \item (8 marks) Consider a source and destination pair connected by a network
        path of capacity 100 Mbps and a round-trip time of 100 ms (measured from
        when the source transmits the first bit of a packet until the destination’s
        ack is received). Assume a packet size of 1000 bytes, and that the loss
        probability is negligibly small.
    \begin{enumerate}
        \item Give the maximum achievable data transfer rate in Mbps, assuming
            use of the stop- and-wait reliable data transfer protocol.
        \item Give the maximum achievable data transfer rate in Mbps, assuming
            use of a sliding window reliable data transfer protocol with maximum
            sending window sizes of 10 and 1000.
        \item Assuming again use of a sliding window protocol, how big would the
            maximum sending window have to be to get a maximum achievable data
            transfer rate equal to the network path capacity of 100 Mbps?
        \item Under what circumstances, given these values for capacity,
            round-trip time, and packet size, could it be beneficial to have a
            maximum sending window size larger than that given in your answer to
            part (c)?
    \end{enumerate}
\end{enumerate}


\end{document}
